% Options for packages loaded elsewhere
\PassOptionsToPackage{unicode}{hyperref}
\PassOptionsToPackage{hyphens}{url}
%
\documentclass[
]{article}
\usepackage{lmodern}
\usepackage{amsmath}
\usepackage{ifxetex,ifluatex}
\ifnum 0\ifxetex 1\fi\ifluatex 1\fi=0 % if pdftex
  \usepackage[T1]{fontenc}
  \usepackage[utf8]{inputenc}
  \usepackage{textcomp} % provide euro and other symbols
  \usepackage{amssymb}
\else % if luatex or xetex
  \usepackage{unicode-math}
  \defaultfontfeatures{Scale=MatchLowercase}
  \defaultfontfeatures[\rmfamily]{Ligatures=TeX,Scale=1}
\fi
% Use upquote if available, for straight quotes in verbatim environments
\IfFileExists{upquote.sty}{\usepackage{upquote}}{}
\IfFileExists{microtype.sty}{% use microtype if available
  \usepackage[]{microtype}
  \UseMicrotypeSet[protrusion]{basicmath} % disable protrusion for tt fonts
}{}
\makeatletter
\@ifundefined{KOMAClassName}{% if non-KOMA class
  \IfFileExists{parskip.sty}{%
    \usepackage{parskip}
  }{% else
    \setlength{\parindent}{0pt}
    \setlength{\parskip}{6pt plus 2pt minus 1pt}}
}{% if KOMA class
  \KOMAoptions{parskip=half}}
\makeatother
\usepackage{xcolor}
\IfFileExists{xurl.sty}{\usepackage{xurl}}{} % add URL line breaks if available
\IfFileExists{bookmark.sty}{\usepackage{bookmark}}{\usepackage{hyperref}}
\hypersetup{
  pdftitle={Untitled},
  pdfauthor={me},
  hidelinks,
  pdfcreator={LaTeX via pandoc}}
\urlstyle{same} % disable monospaced font for URLs
\usepackage[margin=1in]{geometry}
\usepackage{graphicx}
\makeatletter
\def\maxwidth{\ifdim\Gin@nat@width>\linewidth\linewidth\else\Gin@nat@width\fi}
\def\maxheight{\ifdim\Gin@nat@height>\textheight\textheight\else\Gin@nat@height\fi}
\makeatother
% Scale images if necessary, so that they will not overflow the page
% margins by default, and it is still possible to overwrite the defaults
% using explicit options in \includegraphics[width, height, ...]{}
\setkeys{Gin}{width=\maxwidth,height=\maxheight,keepaspectratio}
% Set default figure placement to htbp
\makeatletter
\def\fps@figure{htbp}
\makeatother
\setlength{\emergencystretch}{3em} % prevent overfull lines
\providecommand{\tightlist}{%
  \setlength{\itemsep}{0pt}\setlength{\parskip}{0pt}}
\setcounter{secnumdepth}{-\maxdimen} % remove section numbering
\usepackage{booktabs}
\ifluatex
  \usepackage{selnolig}  % disable illegal ligatures
\fi

\title{Untitled}
\author{me}
\date{2022-07-16}

\begin{document}
\maketitle

\hypertarget{regressao-original-com-lead}{%
\section{Regressao original com
lead}\label{regressao-original-com-lead}}

\begin{table}[htbp]
\centering
\caption{TABELA A5.2: Impacto do clima no número de feminicídio do Brasil com lead}
\begin{tabular}{lccc}
\tabularnewline\midrule\midrule
Dependent Variable:&\multicolumn{3}{c}{feminicidio}\\
Model:&(1) & (2) & (3)\\
 &  OLS & Poisson & Neg. Bin.\\
\midrule \emph{Variables}&   &   &  \\
Temperatura 3.6° - 13.2° & -0.0008 & -0.0511 & -0.0652\\
  &(0.0016) & (0.0939) & (0.0920)\\
Temperatura 13.2° - 15.1° & 0.0012 & 0.0709 & 0.0644\\
  &(0.0015) & (0.0801) & (0.0786)\\
Temperatura 15.1° - 16.6° & 0.0007 & 0.0464 & 0.0351\\
  &(0.0012) & (0.0684) & (0.0664)\\
Temperatura 16.6° - 18° & $2.04\times 10^{-5}$ & -0.0102 & -0.0124\\
  &(0.0010) & (0.0547) & (0.0532)\\
Temperatura 18° - 19.5° & $5.6\times 10^{-5}$ & -0.0069 & -0.0125\\
  &(0.0008) & (0.0390) & (0.0386)\\
Temperatura 21° - 22.4° & -0.0003 & -0.0243 & -0.0252\\
  &(0.0008) & (0.0412) & (0.0419)\\
Temperatura 22.4° - 24.1° & -0.0006 & -0.0318 & -0.0416\\
  &(0.0012) & (0.0631) & (0.0629)\\
Temperatura 24.1° - 25.6° & -0.0005 & -0.0212 & -0.0293\\
  &(0.0014) & (0.0790) & (0.0790)\\
Temperatura 25.6° - 31.5° & -0.0017 & -0.1158 & -0.1253\\
  &(0.0016) & (0.0922) & (0.0926)\\
Precipitação 6.5mm - 22.6mm & 0.0006 & 0.0359 & 0.0257\\
  &(0.0004) & (0.0297) & (0.0296)\\
Precipitação 22.6mm - 56mm & -0.0007 & -0.0488 & -0.0535\\
  &(0.0005) & (0.0332) & (0.0336)\\
Precipitação 56mm - 115.4mm & 0.0002 & $9.54\times 10^{-5}$ & -0.0109\\
  &(0.0006) & (0.0375) & (0.0376)\\
Precipitação 115.4mm - 694.8mm & 0.0007 & 0.0326 & 0.0199\\
  &(0.0008) & (0.0437) & (0.0439)\\
\midrule \emph{Fixed-effects}&   &   &  \\
data & Yes & Yes & Yes\\
code\_muni & Yes & Yes & Yes\\
\midrule \emph{Fit statistics}&  & & \\
Observations & 1,102,068&654,984&654,984\\
Squared Correlation & 0.21904&0.21314&0.21053\\
Pseudo R$^2$ & -0.28096&0.23784&0.19707\\
BIC & -1,161,853.2&180,983.4&180,662.9\\
Over-dispersion & &&2.3012\\
\midrule\midrule\multicolumn{4}{l}{\emph{Clustered (code\_muni) standard-errors in parentheses}}\\
\multicolumn{4}{l}{\emph{Signif. Codes: ***: 0.01, **: 0.05, *: 0.1}}\\
\end{tabular}

\medskip \emph{Notes:} As colunas (1), (2) e (3) são referentes ao modelo simples de mínimos quadrados ordinários, Poisson e binomial negativa, respectivamente, com efeito fixo de região e tempo mês-ano e lead de (t + 1). Os espectros de temperatura e precipitação foram divididos em decil e quintil, respectivamente. Os intervalos [19.5°C – 21°C) e [0 mm - 6.7 mm) são usados como padrão nas dummies de temperatura e precipitação, respectivamente. Significância: ***p<0.01, **p<0.05, *p<0.1.
\end{table}

\hypertarget{regressao-original-com-lag}{%
\section{Regressao original com lag}\label{regressao-original-com-lag}}

\begin{table}[htbp]
\centering
\caption{TABELA A5.1: Impacto do clima no número de feminicídio do Brasil com lag}
\begin{tabular}{lccc}
\tabularnewline\midrule\midrule
Dependent Variable:&\multicolumn{3}{c}{feminicidio}\\
Model:&(1) & (2) & (3)\\
 &  OLS & Poisson & Neg. Bin.\\
\midrule \emph{Variables}&   &   &  \\
Temperatura 3.6° - 13.1° & 0.0003 & 0.0182 & 0.0296\\
  &(0.0014) & (0.0801) & (0.0825)\\
Temperatura 13.1° - 15° & -0.0001 & -0.0118 & -0.0095\\
  &(0.0012) & (0.0666) & (0.0698)\\
Temperatura 15° - 16.6° & 0.0005 & 0.0234 & 0.0206\\
  &(0.0010) & (0.0554) & (0.0586)\\
Temperatura 16.6° - 18° & 0.0002 & -0.0038 & 0.0025\\
  &(0.0008) & (0.0443) & (0.0469)\\
Temperatura 18° - 19.5° & 0.0004 & 0.0093 & 0.0016\\
  &(0.0008) & (0.0354) & (0.0369)\\
Temperatura 21° - 22.4° & 0.0010 & 0.0555 & 0.0449\\
  &(0.0008) & (0.0417) & (0.0418)\\
Temperatura 22.4° - 24.1° & 0.0010 & 0.0675 & 0.0582\\
  &(0.0013) & (0.0711) & (0.0683)\\
Temperatura 24.1° - 25.6° & 0.0012 & 0.0802 & 0.0655\\
  &(0.0016) & (0.0919) & (0.0890)\\
Temperatura 25.6° - 31.5° & 0.0008 & 0.0475 & 0.0279\\
  &(0.0018) & (0.1044) & (0.1019)\\
Precipitação 6.8mm - 23.7mm & 0.0002 & 0.0081 & 0.0128\\
  &(0.0004) & (0.0297) & (0.0304)\\
Precipitação 23.7mm - 57.5mm & 0.0006 & 0.0311 & 0.0303\\
  &(0.0005) & (0.0328) & (0.0330)\\
Precipitação 57.5mm - 116.1mm & 0.0005 & 0.0352 & 0.0345\\
  &(0.0006) & (0.0373) & (0.0374)\\
Precipitação 116.1mm - 694.8mm & 0.0002 & 0.0202 & 0.0154\\
  &(0.0007) & (0.0436) & (0.0432)\\
\midrule \emph{Fixed-effects}&   &   &  \\
data & Yes & Yes & Yes\\
code\_muni & Yes & Yes & Yes\\
\midrule \emph{Fit statistics}&  & & \\
Observations & 1,102,068&663,498&663,498\\
Squared Correlation & 0.21868&0.21189&0.20899\\
Pseudo R$^2$ & -0.28226&0.23783&0.19728\\
BIC & -1,155,117.7&182,655.8&182,330.8\\
Over-dispersion & &&2.2831\\
\midrule\midrule\multicolumn{4}{l}{\emph{Clustered (code\_muni) standard-errors in parentheses}}\\
\multicolumn{4}{l}{\emph{Signif. Codes: ***: 0.01, **: 0.05, *: 0.1}}\\
\end{tabular}

\medskip \emph{Notes:} As colunas (1), (2) e (3) são referentes ao modelo simples de mínimos quadrados ordinários, Poisson e binomial negativa, respectivamente, com efeito fixo de região e tempo mês-ano e lag de (t - 1). Os espectros de temperatura e precipitação foram divididos em decil e quintil, respectivamente. Os intervalos [19.5°C – 21°C) e [0 mm - 6.7 mm) são usados como padrão nas dummies de temperatura e precipitação, respectivamente. Significância: ***p<0.01, **p<0.05, *p<0.1.
\end{table}

\hypertarget{regressao-original-com-lead-100000}{%
\section{Regressao original com lead
+100000}\label{regressao-original-com-lead-100000}}

\begin{table}[htbp]
\centering
\caption{TABELA A13: Impacto do clima no número de feminicídio do Brasil com lead para municípios acima de 100 mil habitantes}
\begin{tabular}{lccc}
\tabularnewline\midrule\midrule
Dependent Variable:&\multicolumn{3}{c}{feminicidio}\\
Model:&(1) & (2) & (3)\\
 &  OLS & Poisson & Neg. Bin.\\
\midrule \emph{Variables}&   &   &  \\
Temperatura 7.2° - 13.2° & $9.63\times 10^{-5}$ & 0.0112 & 0.0261\\
  &(0.0207) & (0.1317) & (0.1306)\\
Temperatura 13.2° - 14.6° & 0.0147 & 0.1164 & 0.1347\\
  &(0.0185) & (0.1179) & (0.1154)\\
Temperatura 14.6° - 15.7° & 0.0076 & 0.0680 & 0.0743\\
  &(0.0149) & (0.0946) & (0.0943)\\
Temperatura 15.7° - 16.7° & 0.0160 & 0.1337$^{*}$ & 0.1386$^{*}$\\
  &(0.0132) & (0.0798) & (0.0783)\\
Temperatura 16.7° - 18.1° & -0.0047 & -0.0348 & -0.0278\\
  &(0.0118) & (0.0707) & (0.0672)\\
Temperatura 19.6° - 21.3° & -0.0034 & -0.0033 & 0.0085\\
  &(0.0152) & (0.0656) & (0.0611)\\
Temperatura 21.3° - 23° & -0.0046 & -0.0141 & -0.0101\\
  &(0.0166) & (0.0767) & (0.0753)\\
Temperatura 23° - 25.4° & 0.0035 & 0.0352 & 0.0489\\
  &(0.0254) & (0.1202) & (0.1181)\\
Temperatura 25.4° - 31.4° & -0.0112 & -0.0401 & -0.0300\\
  &(0.0269) & (0.1301) & (0.1306)\\
Precipitação 10.9mm - 32.9mm & 0.0062 & 0.0464 & 0.0363\\
  &(0.0069) & (0.0479) & (0.0478)\\
Precipitação 33mm - 64.9mm & -0.0030 & -0.0123 & -0.0200\\
  &(0.0071) & (0.0505) & (0.0511)\\
Precipitação 64.9mm - 119.7mm & 0.0077 & 0.0544 & 0.0472\\
  &(0.0083) & (0.0551) & (0.0562)\\
Precipitação 119.7mm - 651.8mm & 0.0076 & 0.0562 & 0.0434\\
  &(0.0098) & (0.0636) & (0.0643)\\
\midrule \emph{Fixed-effects}&   &   &  \\
data & Yes & Yes & Yes\\
code\_muni & Yes & Yes & Yes\\
\midrule \emph{Fit statistics}&  & & \\
Observations & 54,494&50,314&50,314\\
Squared Correlation & 0.28308&0.28592&0.28177\\
Pseudo R$^2$ & 0.22319&0.20569&0.16182\\
BIC & 71,875.5&47,413.9&47,309.9\\
Over-dispersion & &&4.2178\\
\midrule\midrule\multicolumn{4}{l}{\emph{Clustered (code\_muni) standard-errors in parentheses}}\\
\multicolumn{4}{l}{\emph{Signif. Codes: ***: 0.01, **: 0.05, *: 0.1}}\\
\end{tabular}

\medskip \emph{Notes:} As colunas (1), (2) e (3) são referentes ao modelo simples de mínimos quadrados ordinários, Poisson e binomial negativa, respectivamente, com efeito fixo de região e tempo mês-ano e lead de (t + 1). Os espectros de temperatura e precipitação foram divididos em decil e quintil, respectivamente. Os intervalos [18.1°C – 19.6°C) e [0 mm - 10.9 mm) são usados como padrão nas dummies de temperatura e precipitação, respectivamente. Significância: ***p<0.01, **p<0.05, *p<0.1.
\end{table}

\hypertarget{regressao-original-com-lag-100000}{%
\section{Regressao original com lag
+100000}\label{regressao-original-com-lag-100000}}

\begin{table}[htbp]
\centering
\caption{TABELA A14: Impacto do clima no número de feminicídio do Brasil com lag para municípios acima de 100 mil habitantes}
\begin{tabular}{lccc}
\tabularnewline\midrule\midrule
Dependent Variable:&\multicolumn{3}{c}{feminicidio}\\
Model:&(1) & (2) & (3)\\
 &  OLS & Poisson & Neg. Bin.\\
\midrule \emph{Variables}&   &   &  \\
Temperatura 7.2° - 13.2° & -0.0186 & -0.1018 & -0.0859\\
  &(0.0176) & (0.1182) & (0.1195)\\
Temperatura 13.2° - 14.6° & -0.0152 & -0.0892 & -0.0776\\
  &(0.0147) & (0.0952) & (0.1000)\\
Temperatura 14.6° - 15.6° & -0.0066 & -0.0226 & -0.0358\\
  &(0.0137) & (0.0884) & (0.0894)\\
Temperatura 15.6° - 16.7° & -0.0036 & -0.0088 & -0.0023\\
  &(0.0101) & (0.0679) & (0.0724)\\
Temperatura 16.7° - 18° & -0.0040 & -0.0265 & -0.0091\\
  &(0.0092) & (0.0551) & (0.0567)\\
Temperatura 19.6° - 21.3° & 0.0002 & 0.0108 & 0.0143\\
  &(0.0092) & (0.0416) & (0.0447)\\
Temperatura 21.3° - 23° & 0.0086 & 0.0571 & 0.0578\\
  &(0.0139) & (0.0680) & (0.0710)\\
Temperatura 23° - 25.4° & 0.0074 & 0.0538 & 0.0582\\
  &(0.0246) & (0.1238) & (0.1231)\\
Temperatura 25.4° - 31.4° & -0.0053 & -0.0218 & -0.0211\\
  &(0.0279) & (0.1415) & (0.1417)\\
Precipitação 11.7mm - 34.6mm & 0.0051 & 0.0432 & 0.0397\\
  &(0.0058) & (0.0412) & (0.0420)\\
Precipitação 34.6mm - 66.9mm & 0.0085 & 0.0713 & 0.0715\\
  &(0.0076) & (0.0530) & (0.0529)\\
Precipitação 66.9mm - 120.7mm & 0.0162$^{**}$ & 0.1141$^{**}$ & 0.1111$^{**}$\\
  &(0.0078) & (0.0534) & (0.0550)\\
Precipitação 120.7mm - 651.8mm & 0.0087 & 0.0650 & 0.0605\\
  &(0.0089) & (0.0591) & (0.0603)\\
\midrule \emph{Fixed-effects}&   &   &  \\
data & Yes & Yes & Yes\\
code\_muni & Yes & Yes & Yes\\
\midrule \emph{Fit statistics}&  & & \\
Observations & 54,494&50,567&50,567\\
Squared Correlation & 0.28436&0.28770&0.28297\\
Pseudo R$^2$ & 0.22356&0.20639&0.16249\\
BIC & 72,079.9&47,521.1&47,417.8\\
Over-dispersion & &&4.2781\\
\midrule\midrule\multicolumn{4}{l}{\emph{Clustered (code\_muni) standard-errors in parentheses}}\\
\multicolumn{4}{l}{\emph{Signif. Codes: ***: 0.01, **: 0.05, *: 0.1}}\\
\end{tabular}

\medskip \emph{Notes:} As colunas (1), (2) e (3) são referentes ao modelo simples de mínimos quadrados ordinários, Poisson e binomial negativa, respectivamente, com efeito fixo de região e tempo mês-ano e lag de (t - 1). Os espectros de temperatura e precipitação foram divididos em decil e quintil, respectivamente. Os intervalos [18.1°C – 19.6°C) e [0 mm - 11.4 mm) são usados como padrão nas dummies de temperatura e precipitação, respectivamente. Significância: ***p<0.01, **p<0.05, *p<0.1.
\end{table}

--\textgreater{}

--\textgreater{} --\textgreater{} --\textgreater{} --\textgreater{}
--\textgreater{} --\textgreater{} --\textgreater{} --\textgreater{}
--\textgreater{}

--\textgreater{}

--\textgreater{}

--\textgreater{} --\textgreater{} --\textgreater{} --\textgreater{}
--\textgreater{} --\textgreater{} --\textgreater{} --\textgreater{}
--\textgreater{} --\textgreater{}

--\textgreater{}

--\textgreater{} --\textgreater{} --\textgreater{} --\textgreater{}
--\textgreater{} --\textgreater{} --\textgreater{} --\textgreater{}
--\textgreater{} --\textgreater{}

\end{document}
